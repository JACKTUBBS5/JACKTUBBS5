
\documentclass{interact}


%++++++++++++++++++++++
% added packages

\usepackage{ graphicx, rotating, float}

\usepackage{setspace}
\usepackage[numbers,sort&compress]{natbib}
\bibpunct[, ]{[}{]}{,}{n}{,}{,}
\renewcommand\bibfont{\fontsize{10}{12}\selectfont}

%\usepackage{listings, booktabs}
%\usepackage{threeparttable}
\usepackage{color}
\usepackage{array}
%++++++++++++++++++++++
\theoremstyle{definition}
\newtheorem{theorem}{Theorem}
\newtheorem{ex}{Exercise}
\newtheorem*{prf}{Proof}
\newcommand{\R}{\mathbb{R}}
\newcommand{\N}{\mathbb{N}}
\newcommand{\A}{\textbf{\textit{A}}}
\newcommand{\B}{\textbf{\textit{B}}}
\newcommand{\X}{\textbf{\textit{X}}}
\newcommand{\Y}{\textbf{\textit{Y}}}
\newcommand{\M}{\textbf{\textit{M}}}
\newcommand{\C}{\textbf{\textit{C}}}
\newcommand{\V}{\textbf{\textit{V}}}
\newcommand{\D}{\textbf{\textit{D}}}
\newcommand{\I}{\textbf{\textit{I}}}
\newcommand{\ROC}{\text{ROC}}
\newcommand{\Om}{\textbf{\textit{O}}}
\newcommand{\Q}{\textbf{\textit{Q}}}
\newcommand{\Z}{\textbf{\textit{Z}}}
\newcommand{\Sm}{\textbf{\textit{S}}}
\newcommand{\Lc}{\boldsymbol{\Lambda}}
\newcommand\barbelow[1]{\stackunder[1.2pt]{$#1$}{\rule{.8ex}{.075ex}}}





\newcolumntype{M}{>{\raggedright \arraybackslash}m{12cm}}
\newcommand{\auc}{\text{AUC}}

\graphicspath{{./Figures/}}



\articletype{ARTICLE TEMPLATE}

\begin{document}

     \title{Beta Regression for Modeling the ROC as a function of\\ Continuous Covariates}
\author{
\name{Sarah Stanley\textsuperscript{a} \thanks{CONTACT S. Stanley. Email: sarah\_stanley@baylor.edu}  and Jack~D. Tubbs\textsuperscript{b}\thanks{\color{blue}{I'm making suggests in this version - use as suggestions only - rewrite as you see best}} }
\affil{Department of Statistical Science, Baylor University, Waco, Texas, USA}
}


\maketitle

\begin{abstract} 
      The receiver operating characteristic (ROC) curve is a well-accepted measure of accuracy for diagnostic tests.  In many applications, a test's performance is affected by covariates. As a result,
%which should be accounted for in analysis.  
several regression methodologies have been developed to model the ROC as a function of covariate effects within the generalized linear model (GLM) framework.   In this article, we present an alternative  to two existing,  a parametric and semi-parametric, methods for estimating a covariate adjusted ROC. 
The parametric and semi-parametric methods utilize generalized linear models for binary data who's expected value is
% as dependent upon 
%using binary indicators based on placement values which represent 
the probability that the test result for a diseased subject exceeds that of a non-diseased subject with the same covariate values.   This probability is referred to as the placement value.
%A consequence of using binary indicators in this way is additional correlation in the model that must be accounted for through methods such as bootstrapping.  As an alternative, we propose a 
The new method directly models the placement values. 
%distribution of the placement values directly through beta regression.  Given that the placement values are independent probabilities, a direct beta model is easily implemented and avoids the additional correlation in the previously mentioned methods.   
The proposed method is compared with the existing models with simulation and two clinical studies.
% and show that the new method yields comparable ROC estimates without inducing additional correlation.
     \end{abstract}


\begin{keywords}
Placement values
%AUC regression, Jonckheere-Terpstra trend test, closed test, family-wise error rate, minimum effective dose
\end{keywords}


\section{Introduction}

%Our goal is to investigate
A long standing problem in the testing literature is to determine and control how  covariates affect a test's ability to distinguish between two,  non-diseased and diseased,  populations.  The receiver operating characteristic (ROC) curve is commonly used as a  measure of accuracy for diagnostic tests.
% by implementing an ROC regression procedure.  
Pepe(1998) provides a review of three major  approaches to ROC regression. In this article, we focus on the approach %that the third of which 
that directly models the ROC as opposed to;  modeling the underlying distributions of test response for the diseased and non-diseased populations.  Advantages to this approach include; the accommodation of multiple test types, use of continuous covariates, and the ability to restrict the model to the portion of the ROC that is of interest.  When originally proposed, Pepe's approach was difficult to implement, but simplifications have been made.  In particular, Pepe(2000) proposed a generalized linear model framework for the ROC given by 

\begin{equation}\label{eq1} ROC_X(t) = g(h_0(t)) + X'\beta),\end{equation} 
for $ t \in (0,1)$ where g is a monotone link function, $X$ is a vector of covariates, $h_0(\cdot)$ is a monotonic increasing function and $\beta$ is a vector of the model parameters.\\ % Note that in the ROC-GLM context, the dependent variable is not directly observable.  We thus, write the ROC curve in terms of placement values and view the ROC as the cdf of the placement values or as the conditional expectation of the binary indicator $B_{Dt} = I[PV_D \leq t].$

 Alonzo and Pepe(2002) further increased the utility of (\ref{eq1}) by specifying a parametric form for $h_0(\cdot)$ and using a binary indicator as an outcome variable.  Thus, rather than perform pairwise comparisons between each observation (as in the Mann-Whitney statistics), Alonzo and Pepe compared each diseased observation to a specified set of covariate-adjusted quantiles for the non-diseased population. The resultant binary value could then be modeled using  a logistic regression approach. 
 %, thereby substantially reducing the amount of computation.  

{\color{blue} I don't think placement value when considering Alonzo. Should we save this for the next section? In addition to the reduced number of comparisons, the advantages of Alonzo and Pepe's method include a simplified conceptual framework in which one can interpret the ROC in terms of placement values.  A placement value is a right hand probability that can be measured by the survival function.  Thus, for a diseased observation $Y_D,$ we have that the placement value in terms of the reference survival function is defined as $ PV_D = S_{\bar{D}}(Y_D). $  That is, given a diseased observation ($Y_D),$ the placement value is found by mapping $Y_D$ onto the reference population and finding the probability that a randomly selected reference individual will have a test response greater than $Y_D.$  The ROC curve is thus equivalent to the cdf of the diseased placement values $PV_D.$  We have
\begin{align*}
		P[PV_D \leq t | X] &= P[S_{\bar{D}X}(Y_D) \leq t |X] \\
		&=  P[Y_D \geq [S_{\bar{D}X}^{-1}(t) | X]\\
		&= ROC_{X}(t).
		\end{align*}  
}		
		
%The idea of interpreting the ROC as the distribution of placement values was extended by 
Cai(2004) proposed a semi-parametric model by demonstrating that (\ref{eq1}) is equivalent to $h_0(PV_D) = - X'\beta + \epsilon,$ where $h_0(\cdot)$ is unknown.
%semi-parametric model.
% in which the parametric distribution assumption on $h_0$ was relaxed. 
Implementation of this model is dependent upon pairwise comparisons of the placement value, given by $PV_D = \Pr[ Y_D > Y_{\bar D} | X],$ to estimate the covariate effects $\beta$ that are then included as an offset in the estimation of $h_0.$ \\ 

%Note that in each of the parametric and semiparametric approaches discussed above, estimation of the ROC involves the use of binary comparions.  The resulting standard errors are thus incorrect and a bootstrapping procedure (or similar method) must be implemented to account for the added correlation.  Recall however that the ROC can be written as the cdf of the placement values which make up a set of independent probabilities.  As an alternative to the pre-existing procecdures, we thus propose the direct modeling of the placement values via beta regression, thereby eliminating the need for binary comparisons and avoiding additional correlation in the model.  \\

The outline for this article is as follows.  In section 2, we describe in greater detail the three models considered in this article. 
%on each of the parametric, semiparametric, and beta approaches.  
Section 3 contains simulation results comparing the performances of the three methods.  Section 4 includes a data example, and we conclude with a discussion in section 5.  \\

%\begin{figure}[!ht]
%\begin{center}
%	  \includegraphics[scale = .4]{PV_220_colored.png}\end{center}
%\end{figure}
\section{Methods}
In this section, two existing methods are briefly discussed before we present a new method. Each of the methods make use of a term defined by Pepe (???), called the placement value.  For completeness, we provide a brief discussion of placement vales.
\subsection{Placement Values - $PV_D$}
  A placement value is a right hand probability that can be measured by the survival function.  Thus, for a diseased observation $Y_D,$ we have that the placement value in terms of the reference survival function is defined as $ PV_D = S_{\bar{D}}(Y_D). $  That is, given a diseased observation ($Y_D),$ the placement value is found by mapping $Y_D$ onto the non-diseased population and finding the probability that a randomly selected non-diseased subject will have a test response greater than $Y_D.$  The ROC curve is thus equivalent to the cdf of the  placement values $PV_D,$ where 
\begin{align*}
		\Pr[PV_D \leq t | X] &= \Pr[S_{\bar{D}X}(Y_D) \leq t |X] \\
		&=  \Pr[Y_D \geq S_{\bar{D}X}^{-1}(t) | X]\\
		&= ROC_{X}(t).
		\end{align*}  

\subsection{Parametric Approach}

Our objective is to determine the effect  a covariate  $X$ has on the accuracy of a diagnostic test $Y,$ where larger values of $Y$ indicate disease.  Let $Y_D$ denote the test result for an observation from the  diseased population and $Y_{\bar{D}}$ denote the test result for an observation from  the non-diseased (reference) population.  Suppose that we classify a subject as being from the diseased population  
% by choosing a cut-off value $c$ such that if $Y<c$, the individual is healthy, and 
if $Y \geq c.$ 
%the individual is diseased.  
The  test's 
%probability that the test correctly identifies a subject as diseased is given by the 
true positive rate is, TPR$(c) = \Pr[Y \geq c|D].$ Similarly, the test's 
%probability that a test incorrectly classifies a subject as diseased is given by the
 false positive rate is, FPR(c) = $\Pr[Y \geq c | \bar{D}].$  {\color{blue} go ahead and use the equation that defines the ROC as a function of these two rates. The ROC curve defined as the set of all TPR-FPR pairs quantifies the separation between the diseased and healthy populations.}

Let  $X$ denote covariates common to both populations, such as age and bmi. Let $X_D$ denote covariates that are specific to the diseased group, such as, disease duration, severity or previous treatment.
The ROC  can be written as 
\begin{equation}\label{roc} \ROC_{X, X_D}(t) = S_{D,X,X_D}(S^{-1}_{\bar{D}, X}(t)),\end{equation}
 for $t \in (0,1),$
 % is the FPR.  Note that 
 $S_{D,X,X_D}(c) = P(Y_D \geq c|X, X_D)$ and $S_{\bar{D}, X}(c) = P(Y_{\bar{D}} \geq c|X)$ are survival functions at threshold $c.$ In which case, the  $\ROC_{X, X_D}(t)$ is the probability that a test result, $Y_D,$ for a diseased subject is greater than or equal to the t$^{th}$ quantile 
 %from the distribution of
 for the covariate adjusted test results of non-diseased subjects. 

Alonzo and Pepe (2002) proposed a  parametric extension of (\ref{eq1}) as, 
\begin{equation}\label{alonzo} \ROC_{X, X_D}(t) = g(  \gamma_1 h_1(t) + \gamma_2 h_2(t) + \beta X + \beta_D X_D),
\end{equation} where  $h_1(t) = 1, h_2(t) ={\Phi}^{-1}(t)$, and $g(\cdot) = {\Phi(\cdot)}$ where ${\Phi(\cdot)}$ is the cdf of the standard normal. Note,  this approach is known as a parametric distribution free method because a parametric model is specified for the ROC, but no assumptions are made about the distributions of the test results $Y_D$ and $Y_{\bar{D}}.$  %The model is called the binormal model.

The parametric model, (\ref{alonzo}), follows from Pepe(2000) where the ROC is written as the expectation of the binary indicator $U_{ij} = I[Y_{D_i} \geq Y_{\bar{D}_j}]$ for all pairs of observations $\{(Y_{D_i}, Y_{\bar{D}_j}), i = 1,\ldots, n_D; j = 1,\ldots, n_{\bar{D}} \},$ with $n_D$ and $n_{\bar{D}}$ denoting the number of observations from the diseased and reference populations, respectively.  
%To ease computation, 
Alonzo and Pepe(2002) proposed a modification by replacing $Y_{\bar{D}_j}$ with  $S^{-1}_{\bar{D}, X_i}(t)$, for $t \in T = \{n_T \mbox { chosen values of FPRs } \in (0,1)\}$. In which case, the binary indicator becomes $U_{it} = I[Y_{D_i} \geq S^{-1}_{\bar{D}, X_i}(t)]$.  
%Thus, Alonzo and Pepe are making $n_D$ to $t$ comparisons rather than $n_D$ to $n_{\bar{D}}$ comparisons.
{\color{blue} the benefit of this method is not the computation but rather the ability to adjust the reference population with covariates, one can't do this with the observed $Y_{\bar D}$.}  
Note, one can write  $$U_{it} = I[Y_{D_i} \geq S^{-1}_{\bar{D}, X_i}(t)] = \Pr[S_{\bar{D}, X_i}(Y_D) \leq t] = \Pr[PV_D \leq t],$$ where $PV_D$ is the placement value for the observation $Y_D$ given the covariate vector $X.$  An algorithm for (\ref{alonzo}) can be written as
\begin{enumerate}
	\item Specify a set $T = \{t_{\ell}: \ell = 1, ..., n_T\} \in (0,1)$ of FPRs.

		\item Estimate the covariate specific survival function $S_{\bar{D}{X}}$ for the reference population at each $t \in T$ using quantile regression.
		
	\item For each diseased observation $y_{D_j}$, calculate the placement values $PV_j = \hat{S}_{\bar{D} \mathcal{X}_{D_j}} (y_{D_j}).$ 
	
	
	\item Calculate the binary placement value indicator $\hat{B}_{jt} = I[PV_j \leq t], t \in T, j = 1,\ldots, n_D.$ 
	\item Fit the model $E[\hat{B}_{jt}] = g^{-1} \bigg( \sum_{k = 1}^K \alpha_k h_k(t) + X'{\beta} \bigg).$
	\end{enumerate}

\noindent In step (1), we specify a set of $n_T$ FPRs, $T$, that are usually equally spaced.  Recall,  the ROC can be summarized by the AUC which is an extension of the Mann Whitney statistic formed by making $n_d$ to $n_{\bar{d}}$ comparisons.  In the parametric approach, Alonzo and Pepe  make $n_d$ to $n_T$ comparisons. 
%thereby reducing the amount of computation.  
In step (2), we estimate the covariate adjusted survival curve for the reference group using quantile regression from which we obtain a non-diseased marker for each $t \in T.$  This set of $n_T$
markers %that are functions of the covariates, which 
will be used in step (3) to calculate the placement values.  Estimation of the reference survival curve is illustrated in Figure 1. {\color{blue} I didn't include graph since I didn't have it} \\

%\begin{figure}[h!]
%		\begin{center}
%	\includegraphics[scale = .3]{quant_surv_928_edited.png}
%		\end{center}
%		\caption{Illustration of step 2.  For each value of $t$ we obtain a predicted value $x$ via quantile regression.  %The set of predicted values provides an estimate of the covariate adjusted survival curve for the reference population. {\color{red} BW version + check figure requirements}}
%\end{figure}


To calculate the placement values of the diseased points, recall that for a diseased observation $Y_D$ the placement value is found by evaluating the estimated reference survival curve at $Y_D.$  Thus, in step (3) we determine where each of the $n_d$ diseased points lies in relation to the markers found from the quantile regression and calculate the right hand probability.  After obtaining the $n_d$ placement values, we create a binary indicator $\hat{B}$ by performing $n_d$ to $t$ comparisons between the placement values and the set of FPRs $t$.  
Note,  the ROC can be modeled as the conditional expectation of $B_{D_t} = I[PV_D \leq t].$  Thus, in step (5), we model $E[\hat{B}_{jt}]$ using a probit link to obtain a covariate adjusted estimate of the ROC.


\subsection{Semi-parametric Approach}
Cai and Pepe(2002) proposed a semi-parametric method by allowing an arbitrary non-parametric baseline function $h_0(\cdot)$ in (\ref{eq1}).
%the model $ ROC_x(t) = g(h_0(t)) + X'\beta), t \in (0,1)$ and 
  Their approach required the  simultaneous estimation of $h_0(\cdot)$ and $\beta$.  Cai(2004) introduced a new method of estimating parameters for the semi-parametric model by showing that (\ref{eq1}) is equivalent to $h_0(PV_D) = - X'\beta + \epsilon,$ where $\epsilon$ is a random variable with known distribution $g$ and $h_0(\cdot)$ is an unspecified increasing function.  
%With the establishment of equivalence, 
Cai used  pairwise comparison of placement values to estimate $\beta,$ before estimating  the baseline function $h_0(\cdot).$ 
%which takes the previously obtained $\beta$ estimates into account via an offset term.  
An algorithm for implementing the semi-parametric approach is as follows.

%\singlespacing
	\begin{enumerate}
	\item Specify a set $T = \{t_{\ell}: \ell = 1, ..., n_T\} \in (0,1)$ of FPRs.

		\item Estimate the covariate specific survival function $S_{\bar{D}{X}}$ via quantile regression.
		
	\item Calculate the placement values $PV_j = \hat{S}_{\bar{D} \mathcal{X}_{D_j}} (y_{D_j}).$ 
	
	
	\item Calculate the binary placement value indicator 
	\begin{center}$\hat{B}_{jt} = I[PV_j \leq t], t \in T, j = 1,\ldots, n_D.$ \end{center}

	\item For each pair of observations in $Y_D$, calculate 
	$$ \widehat{PV}_{j \ell} = I[PV_j \leq PV_{\ell}], \text{ and } x_{j \ell} = x_{D_j} - x_{D_\ell}$$ with $j, \ell = 1,\ldots, n_D, j \neq \ell.$
	\item Fit the following GLM without an intercept to estimate $\boldsymbol{\beta}$
$$g(\widehat{PV}) = -\X'\boldsymbol{\beta}. $$ 
\item Estimate $h_0(\cdot)$ using $\hat{\boldsymbol{\beta}}$ and $\hat{B}_{jt}$ as follows
$$ g(E[\hat{B}_{jt}]) = intercept + \text{offset}(\X'\hat{\boldsymbol{\beta}}).$$

	\end{enumerate}

%\onehalfspacing

Note that steps (1) - (4) are identical to those of the parametric method.  The first difference between the two approaches appears in step (5), where we create a second binary indicator describing the relationship between each pair of placement values.  In this step, we also calculate the pairwise differences for each covariate value.  We then fit a GLM without an intercept to the binary indicator created in step 5, adjusting for covariates using the pairwise differences.  From this model, we obtain an estimate for $\beta.$  In step (7), we then estimate $h_0(\cdot)$ by modeling the binary indicator $\hat{B}$ as a function of the intercept and an offset term that accounts for $\hat{\beta}$ from step (6).


\subsection{Beta Approach}
The parametric and semi-parametric approaches to estimating the covariate adjusted ROC given in equation (\ref{eq1}) made use of the binary random variable defined by the placement values of the diseased response as referenced with the non-diseased population. In this section, we present a method for modeling the covariate adjusted ROC with the placement values as opposed to a binary indicator based upon the placement values. Our method makes use of the beta regression models that are readily available with GLM software.

% discussed previously, correlation is introduced in the model when pairwise comparisons are made.  The resulting standard errors are thus incorrect and the additional correlation must be accounted for through a procedure such as bootstrapping.  Recall, however, that the cdf of the placement values from the diseased population is equivalent to the ROC. In addition, the placement values constitute a set of independent probabilities which we can model directly via beta regression thus eliminating the need for binary comparisons and avoiding additional correlation in the model. \\
   

Before describing an algorithm for the proposed method, we briefly introduce the beta regression model as referenced in Ferrari 2004.  Recall that the mean and variance of $Y \sim$ Beta$(a, b)$ are
$$E(Y) = \frac{a}{a + b} \text{ and } Var(Y) = \frac{a b}{(a + b)^2(a+b +1)}.$$
We will define the beta regression model in terms of $\mu = E(Y)$ and a precision parameter $\phi = a + b$ so that the reparameterized beta distribution mean and variance are
$$E(Y) = \mu \text{ and } Var(Y) = \frac{\mu(1 - \mu)}{1 + \phi}.$$ 


\noindent Let $y_1,\ldots, y_n$ be independent random variables from a beta density with mean $\mu_t,$ t = 1,\ldots,n and scale parameter $\phi.$ 
 Then the beta regression model can be written as
$$ g(\mu_t) = \sum_{i = 1}^k x_{ti} \beta_i = \eta_t,$$
where  $\boldsymbol{\beta}$ is a vector of regression parameters, $x_{t1},\ldots, x_{tk}$ are observations on $k$ covariates, and $g$ is a monotonic link function.
 Using the logit link, we have 
$\mu_t = \frac{1}{1 + e^{-x_t'\beta}}.$
We can thus obtain the original parameters a and b from the beta distribution by calculating
$$ \hat{a} = \frac{\hat{\phi}}{1 + e^{-x_t'\hat{\beta}}} \text{ and }  \hat{b} = \hat{\phi} \bigg( 1 - \frac{1}{1 + e^{-x_t'\hat{\beta}}}\bigg). $$

\noindent An algorithm for the beta method can be written as follows.


	\begin{enumerate}
	\item Specify a set $T = \{t_{\ell}: \ell = 1, \ldots, n_T\} \in (0,1)$ of FPRs.

		\item Estimate the covariate specific survival function $S_{\bar{D}{X}}$ via quantile regression.
		
	\item Calculate the placement values $PV_j = \hat{S}_{\bar{D} \mathcal{X}_{D_j}} (y_{D_j}).$ 
	
	\item Perform a beta regression on the placement values to obtain estimates of $\boldsymbol{\beta}$ and $\phi$.
	\item Transform to obtain $a = \mu \phi$ and $b = ( 1 - \mu)\phi$.
	\item Calculate the cdf of the placement values using the Beta(a,b) distribution found above to obtain the ROC and the AUC.
	
	\end{enumerate}



Steps (1) - (3) are identical to the parametric and semi-parametric cases.  In step (4), we model the placement values directly using beta regression to obtain estimates of $\beta$ and $\phi$, instead of calculating a binary indicator.  We then apply a transformation to return to the original beta parameters $a$ and $b$ and calculate the cdf of the placement values using the resulting Beta($a,b$) distribution which yields an estimate for the ROC.

\section{Simulation}

To compare the three presented methods we perform two simulations, one with binormal data and one using the extreme value distribution.  When both populations are normally distributed, exact solutions for the ROC and AUC exist. We have $ROC(t) = {\Phi}(a + b {\Phi}^{-1} (t) ),$ and $ AUC = \Phi \bigg( \frac{a}{\sqrt{1 + b^2}} \bigg),$ where $ a = \frac{\mu_D - \mu_{\bar{D}}}{\sigma_D}, b = \frac{\sigma_{\bar{D}}}{\sigma_D}.$  We are thus able to compare AUC estimates from each of the three presented methods with the truth.  For this example we simulate data from 
$$Y_D = 2 + 4X + \epsilon_D \text{ and } Y_{\bar{D}} = 1.5 + 3X + \epsilon_{\bar{D}},$$ 
where $X \sim U(0,1)$ and $\epsilon_D, \epsilon_{\bar{D}} \sim N(0,1.5^2).$
That is, $Y_D \sim N(2 + 4X, 1.5^2) \text{ and } Y_{\bar{D}} \sim N(1.5 + 3X, 1.5^2).$
Thus, the true AUC at covariate value $X = x_0$ is 
$$AUC(x_0) = \Phi \bigg(\frac{\mu_{D} - \mu_{\bar{D}}}{(\sigma_D^2 + \sigma_{\bar{D}}^2)^{1/2}} \bigg) = \Phi \bigg(\frac{0.5 + x_0}{\sqrt{4.5}}\bigg). $$

We simulate 300 data sets from the binormal scenario presented
above, and summarize the mean of the AUC estimates for each
method at specified covariate values in Table 1. { \color{red} Beyond showing that the means across the three methods are comparable and addressing standard deviations, what should I be emphasizing here?  Does coverage make sense?}\\



\begin{table}[!th]
\begin{center}
\begin{tabular}{lcccccc}
  \hline
 & $x_0 = 0.5$ & $x_0 = 0.6$ & $x_0 = 0.7$ & $x_0 = 0.8$ \\ 
  \hline
Parametric & 0.6791 & 0.7426 & 0.7983 & 0.8456 \\ 
 % AUCap.sd & 0.0374 & 0.0365 & 0.0358 & 0.0345 \\ 
Semiparametric & 0.6642 & 0.7124 & 0.7567 & 0.7964 \\ 
 % AUCs.sd & 0.0358 & 0.0347 & 0.0342 & 0.0339 \\ 
Beta & 0.6699 & 0.7305 & 0.7831 & 0.8274 \\ 
Truth & 0.6813 & 0.6980 & 0.7142 & 0.7300 \\
 % AUCb.sd & 0.0396 & 0.0365 & 0.0343 & 0.0324 \\ 
   \hline
\end{tabular}
\caption{Mean AUC estimates across 300 data sets for each method}
\end{center}
\end{table}


%\begin{figure}[!th]
%\centering
%\includegraphics[scale = .31]{Binormal_ROC_plot_3b}
%\caption{ Comparison of ROC estimates for each method }
%\end{figure}



%\clearpage

\noindent {\color{red}Extreme Value Results}
\section{Example}
{\color{red} Alonzo example?

DME Protocol I???}

\section{Discussion}

   Given the broad acceptance of the ROC curve as a measure of accuracy for diagnostic tests, our intent was to investigate the effect of covariates on a test's performance through ROC regression.  As noted, several regression methodologies have been developed to model the ROC as a function of covariate effects within the generalized linear model (GLM) framework.  In particular, the parametric and semi-parametric approaches estimate the ROC using binary indicators.  The use of such indicators, however, leads to additional correlation in the model that must be accounted for through methods such as bootstrapping.  In this paper, we proposed a new approach that implements beta regression to model the placement values directly, thereby eliminating the additional correlation induced by the pre-existing methods. We compared our beta methodology with the parametric and semi-parametric approaches via simulation, showing that the new method yields comparable ROC estimates without inducing additional correlation.  
   
\noindent{\color{red}This is currently a rewording of the abstract... more work to be done here.}



\bibliographystyle{tfs}

{\color{blue}\section{Bibliography}Need to create a bibliography for your dissertation that can be used  for all your work} 
\section{References}
{\color{red}\{Transfer to Bibtex  OK\}}
\begin{itemize}
\item Alonzo, T. and M. Pepe (2002), ``Distribution-free ROC analysis using binary regression techniques," \textit{Biostatistics}, 3,  421-432.
\item  Bamber, D. (1975), ``The area above the ordinal
dominance graph and the area below the receiver
operating characteristic graph," \textit{Journal of Mathematical
Psychology}, 12, 387-415.
\item Cai, T. (2004), ``Semi-parametric ROC regression analysis with placement values," \textit{Biostatistics}, 5, 45-60.
\item Ferrari, S. and Cribari-Neto, F. (2004), ``Beta Regression for Modelling Rates and Proportions," \textit{Journal of Applied Statistics}, 31, 799-815.
\item Pepe, M. and T. Cai (2002), ``The analysis of placement values for evaluating discriminatory measures," \textit{UW Biostatistics Working Paper Series.} Working Paper 189. 
\item Rodriguez-Alvarez, M.X. et. al. (2011) ``Comparative Study of ROC regression techniques," \textit{Computational Statistics and Data Analysis,} 55, 888-902.
\end{itemize}

%\bibliography{dissertation_jvz}

%%appendix

\appendix 
\section{Modification of Simulation used in \cite{Zhang_Zhao_Tubbs_2011} }\label{sec: Appendix A}
%\input{Scaling_parameter_simulation}

\cite{Balakrishnan_Nevzorov} derived the true AUC for two treatment groups with standard extreme value error terms. A similar derivation is followed to derive the true AUC when both treatment groups have extreme value distributions with the same scale parameter, $\lambda.$ Let
\begin{eqnarray*}
Y_1 &=& -\frac{1}{\lambda}\log(U_1) + m_1 \cdot x \\
Y_2 &=& -\frac{1}{\lambda}\log(U_2) + m_0 + (m_1+m_2) \cdot x,
\end{eqnarray*}
where
\begin{eqnarray*}
U_1 \sim \text{exp}(1), \quad U_2 \sim \text{exp}(1).
\end{eqnarray*}
We are interested in
\begin{eqnarray*}
\auc(x) &=& P[Y_1 < Y_2] \\
&=& P[ 1/\lambda \log(U_2) - 1/\lambda \log(U_1) < m_0 + m_2 \cdot x] \\
&=& P[ V <  m_0 + m_2 \cdot x ] \\
&=& F_V(m_0 + m_2 \cdot x).
\end{eqnarray*}

In order to find the CDF of V, let $V_1= -1/\lambda \log(U_1).$ Then, $U_1 = \exp(-\lambda V_1).$ It follows that the pdf of $V_1$ is
\begin{equation}
f_{V_1}(v_1) = \exp\{- \exp(-\lambda v_1)\}\cdot \lambda \exp\{- \lambda v_1\} =  \lambda \exp\{- \exp(-\lambda v_1) - \lambda v_1)\},
\end{equation}
where $-\infty < V_1 < \infty$ with $\text{E}(V_1) = 0.57722/\lambda$ and $\text{Var}(V_1)=  \pi^{2}/(6 \cdot \lambda^2).$ The cdf of $V_1$ is
\begin{equation}
F_{V_1}[v_1] = P[V_1 < v_1] = \int_{- \infty} ^ {v_1}   \lambda \exp\{-\lambda \exp(-x) - x)\} \, dx = \exp \{ -\exp (-\lambda v_1).
\end{equation}
Similarly, let $V_2 = -1/\lambda \log(U_2).$ The pdf and cdf of $V_2$ is the same as $V_1.$
We are interested in the distribution of $V = V_1 - V_2$ where $-\infty < V < \infty.$ Define the bivariate transformation
\begin{equation}
V = V_1 - V_2 \quad \text{and} \quad W = V_2.
\end{equation}
That is, $V_1 = V + W$ and $V_2 = W$ with a Jacobian of 1 and $-\infty < V < \infty$ and $-\infty < W < \infty.$
We have that
\begin{eqnarray*}
f_{V_1, V_2}(v_1, v_2) &= \lambda \exp\{- \exp(-\lambda v_1) - \lambda v_1)\} \cdot \lambda \exp\{- \exp(- \lambda v_2) - \lambda v_2)\}\\
\end{eqnarray*}
The CDF of $V$ is derived as follows
\begin{eqnarray*}
F_V(v) &=& \int_{-\infty}^{v} f_v(x) \; dx \\
&=& \int_{-\infty}^{v} \int_{-\infty}^{\infty} f_{V_1, V_2}(x + w, w) \;dw\, dx \\
&=&  \int_{-\infty}^{\infty}\int_{-\infty}^{v} f_{V_1, V_2}(x + w, w) \;dx\, dw \\
&=&  \int_{-\infty}^{\infty} F_{V_1}(v + w)  dF(w) \\
&=&  \int_{-\infty}^{\infty} \exp \{ -\exp [-\lambda(v +w)] \} \cdot \lambda \exp\{- \exp(- \lambda w) - \lambda w \} \; dw \\
&=&  \int_{-\infty}^{\infty} \lambda \exp \{- \exp [- \lambda w] \cdot [1+ \exp  (- \lambda v )] \} \cdot \exp \{ - \lambda w \}  \; dw \\
&=&  \int_{0}^{\infty} \exp \{-u \cdot[1+\exp(- \lambda v)]\}   \; du \\  
&=& \frac{\exp \{ -u[1+\exp(- \lambda v)]\}}{[1+\exp(- \lambda v)]} \Big|_{0}^{\infty} \\
&=& 0 - (-\frac{1}{[1+\exp(- \lambda v)]}) \\
&=& \frac{1}{[1+\exp(- \lambda v)]} \\
\end{eqnarray*}
which is
$$V \sim \text{Logistic} \left( 0, \frac{\pi^{2}}{3*\lambda^2} \right).$$

%This simulation set up was used for the Ingelheim diabetes clinical trial.
 %and the RESTORE and VIVID trials. 
 As an example, suppose that the standard deviation for a treatment group is 132.29. Then the $\lambda$ needed to adjust the standard error of the error structures to fit the standard deviation from the summary statistics is obtained  as 

 %to adjust the standard error of the error structures to fit the standard deviation from the summary statistics of laser treatment group in the RESTORE trial,
% {\color{red} Can you change the example for the Ingelheim diabetes clinical trial?}
 % we have to find the scaling parameter $\lambda$. The standard deviation for this group is 132.29. 
\begin{eqnarray*}
\sqrt{\frac{\pi^{2}}{6*\lambda^2}} = 132.29 \implies \lambda = \sqrt{\frac{\pi^{2}}{6*(132.29)^2}} = 0.0097.
\end{eqnarray*}

%++++++++++++++++++++++++++++++
\section{Competing Methods found in \cite{jan_shieh} and \cite{Shirley}}
\label{sec: Appendix B}
Two existing nonparametric multiple comparison procedures are used as reference for the proposed methods. A brief description of these methods is given.
%\subsection{Shirley's Method \cite {Shirley}}
%\input{shirley_mc}
%shirley_mc
\subsection{Shirley's Multiple Comparison (shi)} \label{shi}
\citep{Shirley}  considered the problem of determining differences in treatment groups that are created by increasing dosage levels of an active compound as compared with a zero-dose control group. The test is a nonparametric version of a parametric procedure given by \cite{Williams}. 

Suppose there are $K$  treatment levels (increasing dosage levels of an active drug) and a zero-dose control group (group 0). \cite{Williams} proposed a procedure based upon the maximum likelihood estimates of the location parameters, $M_i$, subject to the constraint that $M_1 \leq M_2 \leq ... \leq M_K$. The statistic is
$$ t_K=\frac{\hat{M}_K - X_0}{\left(S^2/r_K + S^2/c\right)^{-1/2}}  $$
where $S^2$ is an estimate of the residual variance, $c=r_0$ is the number of observations in the control group and $X_0$ is the control group sample mean. \cite{Williams} provided tables for the critical points for $t_K.$ 

\citep{Shirley} developed a nonparametric version of the Williams test by analyzing the observed ranks instead of the actual data. The results were based on the Wald-Wolfowitz  limit theorem (\cite{wald_1944}), where the vector $\bar{R}=\left(\bar{R}_0, \bar{R}_1, ...\bar{R}_K\right)$ has a limiting multivariate normal distribution and $\bar{R}_i$ is the mean rank of group $i$. The Shirley multiple comparison test is as follows. For equal group sizes, let 
\begin{equation} \label{sheq1}
t = C_{N,K} ~ \left[\underset{1 \leq u \leq K}{\max} \sum_{j=u}^{K} \bar{R}_j (K-u+1)-R_0\right]
\end{equation}
where $C_{N,K} = \left[(K+1)(N+1)/6\right]^{1/2}$ and $N$ is the total sample size. The distribution of $t$ can be approximated by the distribution of $t_K$ when $\nu = \infty$. If the sample sizes are unequal or there are  a considerable number of ties in the data, the statistic becomes
\begin{equation} \label{sheq2}
t = C_{N,K} ~ \left[\underset{1 \leq u \leq K}{\max} \left( \sum_{j=u}^{K} r_j \bar{R}_j /\sum_{j=u}^{K} r_j \right) - \bar{R}_0 \right] 
\end{equation}
where $C_{N,K} = [(N(N+1)/12)(1/r_K + 1/C)]^{1/2}.$ The Shirley multiple comparison test compare each treatment level to the zero-dose control group using either equation (\ref{sheq1}) or (\ref{sheq2}) and the critical points given by \cite{Williams}.

%\subsection{Jan and Shien Method \cite{jan_shieh}}
%\input{janshieh}
\subsection{Jan and Shieh's Multiple Comparison (js)}\label{js}


\cite{jan_shieh} propose a step-down closed testing procedure based on contrasts of the Kruskal-Wallis test to identify the MED. 

The pairwise contrasts are defined within for each of $K+1$ increasing dose levels. Let $Y_{ij}$ denote the response for treatment $i$ and subject $j$. When comparing the $i^{th}$ treatment group to the control group, let $R_{sj}^{(i)}$ denote the rank of $Y_{sj}$ observation within the combination of the first $i$ treatment groups with the control group for $i = 1, \hdots, K$, $s = 0, \hdots, i$, and $j = 1, \hdots, n$. Let $R_{s}^{(i)} = \sum_{j=1}^n R_{sj}^{(i)}$ denote the sum of ranks for the $s^{th}$ dose level. % where $i=1, \hdots, K$, $s=0,1, \hdots, i$ and $j=1, \hdots, n.$ 
A pairwise contrast is defined as $P_i = R_i^{(i)} - R_0^{(i)}$
for $i=1, \hdots, K.$ The proposed statistic to compare the $i^{th}$ dose level to the control is defined as
\begin{equation}\label{eq: js statistic}
Z_i = \frac{P_i}{\sqrt{Var(P_i)}}
\end{equation}
where the null variance of $P_i$ is given by $\text{Var}(P_i) = nN_i(N_i+1)/6$ with $N_i = (i+1)n.$ In the presence of ties, the null variance is adjusted by replacing $N_i+1$ with $N_i + 1 - \sum_{j=1}^{g}t_j(t_j^2-1)/[N_i(N_i-1)].$ Let $\mathbf{Z} = (Z_1, \hdots, Z_{K})'.$ If the global hypothesis hold, then $\mathbf{Z} \sim N_K(\mathbf{0}, \mathbf{R})$ where $\mathbf{R}$ is given by 
$$\mathbf{R} = \begin{bmatrix}
1 & & 1/2 \\
& \ddots & \\
1/2 & & 1
\end{bmatrix}. $$

The MED can be found using the step-down closed testing scheme suggested by \cite{Tamhane}. Let $Z_{i, \rho = 0.5} ^{\alpha}$ denote the upper $\alpha^{th}$ percentile of the multivariate normal distribution with zero mean vector and correlation $\rho = 0.5$. The critical values for $Z_{i, \rho = 0.5} ^{0.05}$ as reproduced from \cite{tamhane_hochberg_1987} are given in Table \ref{tab: js critical values}. %distribution of $Z_{(i)} = \text{max}(JP_1, \hdots, JP_{K})$ for correlation $\rho.$
Let $k_1=K$ and $Z_{(k_1)} = \text{max}(Z_1, \hdots, Z_K).$ Define $d(k_1)$ as the antirank of $Z_{(k_1)}$. That is, $Z_{(k_1)} = Z_{d(k_1)}.$ If $Z_{(k_1)} > Z_{k_1, \rho}^{\alpha}$ then $H_{0i}$ is rejected for $i = d(k_1), \hdots, k_1$. %If $H_{0i}$ is rejected at step 1, let $k_2 = d(k_1)-1.$ 
At the $j^{th}$ step, let $k_j = d(k_{j-1}) -1.$ If $Z_{d(k_j)} > Z_{k_j, \rho=0.5}^{\alpha},$ then reject $H_{0i}$ for $i=d_{k_j}, \hdots, k_j$; otherwise stop testing. When the testing stops at the $m^{th}$ step, then the MED is $k_m+1$.


%\singlespacing
\begin{table}[H] \centering 
  \caption{Critical Values for Jan and Shieh Procedure.} 
  \label{tab: js critical values} 
\begin{tabular}{@{\extracolsep{5pt}}  lccccc} 
\\[-1.8ex]\hline 
\hline \\[-1.8ex] 
i & 1 & 2 & 3 & 4 & 5 \\
\hline \\[-1.8ex] 
$Z_{i, \rho = 0.05}^{0.05}$ &  1.645 & 1.92 & 2.06 & 2.16 & 2.23 \\
\hline \\[-1.8ex] 
\end{tabular} 
\end{table} 
%\doublespacing


\end{document}
